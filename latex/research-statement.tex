%%%%%%%%%%%%%%%%%%%%%%%%%%%%%%%%%%%%%%%%%
% Stylish Curriculum Vitae
% LaTeX Template
% Version 1.0 (18/7/12)
%
% This template has been downloaded from:
% http://www.LaTeXTemplates.com
%
% Original author:
% Stefano (http://stefano.italians.nl/)
%
% IMPORTANT: THIS TEMPLATE NEEDS TO BE COMPILED WITH XeLaTeX
%
% License:
% CC BY-NC-SA 3.0 (http://creativecommons.org/licenses/by-nc-sa/3.0/)
%
% The main font used in this template, Adobe Garamond Pro, does not 
% come with Windows by default. You will need to download it in
% order to get an output as in the preview PDF. Otherwise, change this 
% font to one that does come with Windows or comment out the font line 
% to use the default LaTeX font.
%
%%%%%%%%%%%%%%%%%%%%%%%%%%%%%%%%%%%%%%%%%

\documentclass[a4paper, oneside, final]{scrartcl} % Paper options using the scrartcl class

\usepackage{scrpage2} % Provides headers and footers configuration
\usepackage{titlesec} % Allows creating custom \section's
\usepackage{marvosym} % Allows the use of symbols
\usepackage{tabularx,colortbl} % Advanced table configurations
\usepackage{fontspec} % Allows font customization
\usepackage{ragged2e}
\usepackage{hyperref}
\usepackage{longtable}
\usepackage{ltablex}
\usepackage{pdfcolparcolumns}
\usepackage[gen]{eurosym}
\usepackage{adjustbox}
\usepackage[table]{xcolor}
%\usepackage[round]{natbib}
\usepackage{multibib}


\defaultfontfeatures{Mapping=tex-text}
% \setmainfont{Adobe Garamond Pro} % Main document font

\titleformat{\section}{\large\scshape\raggedright}{}{0em}{}[\titlerule] % Section formatting

\pagestyle{scrheadings} % Print the headers and footers on all pages

\addtolength{\voffset}{-0.5in} % Adjust the vertical offset - less whitespace at the top of the page
\addtolength{\textheight}{3cm} % Adjust the text height - less whitespace at the bottom of the page

\newcommand{\gray}{\rowcolor[gray]{.90}} % Custom highlighting for the work experience and education sections

%separated bibliography
\newcites{OTHER}{References}
\newcites{SELECTED}{Paper submitted as part of the evaluation package}
\newcites{ALL}{Papers not included in the evaluation package}
\newcites{SOFTWARE}{Open-source software references}
\newcites{trash}{foo} % auxiliary cite command
\newcommand\mycite[2]{\citetrash{#1,#2}\nociteSELECTED{#1}\nociteALL{#2}} % new cite command



%----------------------------------------------------------------------------------------
% �FOOTER SECTION
%----------------------------------------------------------------------------------------

\renewcommand{\headfont}{\normalfont\rmfamily\scshape} % Font settings for footer

\cofoot{
\addfontfeature{LetterSpace=20.0}\fontsize{12.5}{17}\selectfont % Letter spacing and font size

Via Morego, 30 {\large\textperiodcentered} 16163 Genova {\large\textperiodcentered} Italy \\ % Your mailing address
{\Large\Letter} francesco.nori@iit.it \ {\Large\Telefon} (+39) 349 66 51 555 % Your email address and phone number
}

%----------------------------------------------------------------------------------------

\begin{document}

%----------------------------------------------------------------------------------------
% �HEADER SECTION
%----------------------------------------------------------------------------------------
\flushleft

{\addfontfeature{LetterSpace=20.0}\fontsize{36}{36}\selectfont\scshape Francesco Nori} % Your name at the top
\\ \vspace{0.5cm}
{\addfontfeature{LetterSpace=20.0}\fontsize{18}{18}\selectfont\scshape Research statement} % Your name at the top

%----------------------------------------------------------------------------------------
%	OBJECTIVE
%----------------------------------------------------------------------------------------
\justifying
\section{Introduction}

\textbf{What:} the focus of my research activities has been on motor control with two fundamental goals: studying humans and implementing humanoids. The achievement of these two goals is guided by two underpinning principles which constitute the backbone of my research activities: a) studies on human motor control should be relevant for building better performing robots; b) implementing humanoids should allow understanding better the human motor control system. The main technological outcome of my research activity is the realisation of humanoids with advanced action and interaction capabilities. The achievement of this goal necessarily passes through advancing the state-of-the-art within the following research areas in robotics: decisional autonomy \citeALL{Berret2012222,delPrete2014}, dependability/adaptability \citeALL{Berret20114354,Nori20071142}, perception \citeALL{Nori2015} and, in a single all-embracing word, cognitive abilities. 

\noindent
\textbf{Where:} my research activity has significantly benefit from the unique interdisciplinary environment available at the Istituto Italiano di Tecnologia. In particular, my work has been deeply rooted in the activities of the 
Robotics, Brain and Cognitive Sciences department (\textsc{Rbcs}) on the one hand and the iCub Facility (\textsc{iCub}) on the other. Connections have been established and will be strengthen with the advanced robotic department (\textsc{Advr}), the pattern analysis and computer vision department (\textsc{Pavis}) and the rehab technologies laboratory (\textsc{Rehab}).  

\noindent
\textbf{Who:} the research achievements described and foreseen in this document  have been and will be possible thanks to the talented and motivated team of collaborators. Our background is in engineering with strong competences in physics, mathematics and computer science. 

\noindent
\textbf{How:} two different tools are at the basis of my research. On the one hand, iCub \textbf{\citeSELECTED{metta2010}} is the technological tool that I personally contributed to develop. On the other hand, control theory is the theoretical toolset which scientifically grounds my technological implementations. In the attempt of combining theoretical and technological development, my research activity produces sound theoretical results validated with dependable implementations on real robots. Adopting a common subdivision in control theory, proposed implementations have required advances of the current state-of-the-art along three different areas: modelling, estimation and control. This subdivision is reflected in the following subsections, which give a systematic organisation to my recent research achievements.

% Within the scope of my research, my activities have advanced the current state-of-the-art in three different macro areas: adaptability (\textbf{modelling}), perception (\textbf{estimation}) and motion/interaction ability (\textbf{control}). In the following, these macro areas have been mapped into their equivalent within the control theory framework: 

\section{Modelling}

Even though humans and humanoids are different in many aspects, part of my research activities aimed at finding abstraction levels at which they can be  modelled with the same mathematical tools. Within this context, dynamics served to model their bodies, optimal control to model movement planning and impedance to model robustness against disturbances. 

\emph{Dynamics.} My research interest has been often focused on dynamics as opposed to kinematics. Dynamics allow to capture quantities, such as forces, which are fundamental for describing both action and interaction. As to this concern, the dynamics of humans and humanoids can be both modelled as articulated rigid bodies \citeOTHER{featherstone07}. Indeed, despite of the underlying simplifications, articulated models have been proven to be quite effective in capturing the mechanics of humanoids \citeOTHER{vukobratovic04} and the biomechanics of humans \citeOTHER{delp2007}. Dynamics can be therefore used as a common modelling tool capable of abstracting from specific differences (e.g. different geometries) while enforcing common mechanical principles (e.g. momentum dynamics). 

\emph{Optimal control.} Inspired by the expressivity of dynamics on the one hand and by the role of optimality \citeOTHER{flash85,todorov2004} on the other, I have been investigating on the role of optimisation and optimal control as a tool for modelling movement planning in both humans and humanoids \citeALL{ivaldi2012}. Leveraging on my previous works on inverse optimal control \citeALL{Nori2004}, I contributed to the problem of modelling arm reaching movements. Human experiments have shown that movement trajectories can be effectively predicted with the combination of multiple cost functions \textbf{\citeSELECTED{berret2011}} while movement variability is best predicted by stochastic optimal control \citeALL{Berret20112086}. 

\emph{Impedance.} In the attempt to find a common principle to explain the role of impedance regulation in humans \citeOTHER{burdet2001} and robots \citeOTHER{hogan1985}, I investigated on the role of muscle co-activation as a mean to achieve robustness in action and interaction planning \citeALL{Berret20133029}. Despite of the significant delays in the human sensory-motor system, muscles seems to possess the interesting property of selectively and passively rejecting disturbances without explicitly resorting on active feedback. Intrigued by this experimental evidence, we were able to mathematically characterise this property \citeALL{Nori2012473} and to design a novel variable stiffness actuator capable of regulating its passive disturbance rejection via agonist-antagonist co-activation \citeALL{Fiorio2012502}. The physical realisation of the design principle in a working prototype \textbf{\citeSELECTED{fiorio14}} posed interestingly theoretical questions which significantly contributed to improving the original design principle itself. Further development of the design are foreseen in the near future. 

\section{Estimation}

In order to improve robots interaction capabilities, advances in their perceptual capabilities are necessary. In particular, robots need to accurately perceive both internal and external forces: the former to control their movements and internal strains, the latter to regulate the interaction with the environment. Inspired by previous literature \citeOTHER{shaeffer2007}, I contributed to the design of a scalable set of sensors to make the iCub capable of regulating internal torques and interaction forces. Evaluating the trade-off between joint level and proximal force/torque sensing \citeALL{Randazzo20114161}, the final adopted solution was the second. The technology allows to efficiently estimate the iCub whole-body dynamics \textbf{\citeSELECTED{fumagalli2012}} exploiting three primary sources of information: (1) embedded force/torque sensors, (2) embedded inertial sensors, (3) distributed tactile sensors (i.e. artificial skin). The methodology relies on a reordered recursive Newton-Euler algorithm, which meets the stringent computational-time constraints necessary for real-time applications \citeSOFTWARE{idyn}. Reframing the estimation problem in a Bayesian framework, the approach has been recently extended to include multiple redundant measurements \citeALL{Nori2015}, while keeping the computational complexity low by exploiting the problem sparsity \citeSOFTWARE{berdy}. 


\section{Control}

Most of my research activities eventually contribute to the final goal of implementing real-time whole-body controllers on the iCub humanoid. Given the iCub complexity, real-time applications call for stringent computational constraints which significantly limit the range of viable implementations. These constraints can only be met by adopting state-of-the-art solutions implemented in computationally optimised software for real-time solving complex non-linear optimisation problems.  

Optimality and computational efficiency are common features for the controllers that I contributed to develop. The cartesian interface \citeALL{Pattacini20101668} is an example of a long-lived and efficient software capable of implementing an optimisation-in-the-loop strategy \citeSOFTWARE{ikin} for performing on-line kinematic inversions. Similarly, the whole-body torque controller \textbf{\citeSELECTED{nori2015a}} implements a state-of-the-art inverse dynamic approach to solve the problem of force regulation in under-actuated but constrained mechanical systems. Simultaneous optimisation and computational efficiency are obtained by reframing the control problem as a quadratic programming optimisation \citeSOFTWARE{torqueBalancing}. Exploiting the internal/external forces estimation strategy described previously, the controller has been successfully adopted in many different balancing tasks: single and double foot support, multiple noncoplanar contacts and constrained goal directed movements.

\section{Research plan}

As opposed to traditional robotic applications which demanded for limited interaction and mobility, robots of the next generations will be required to coordinate physical interaction with physical mobility. Interaction always involve two components: the ``self'' (i.e. the robot) and the ``other'' (i.e. the interacting agent). Successful and  efficient interaction necessarily passes through modelling, estimating and controlling the mutual interaction between the self and the other. 

In a crescendo of complexity, research is expected to cope with increasingly  complicated scenarios. On the one hand, robots (the self) are foreseen to become elastic and compliant. On the other hand, physical interaction is likely to occur not only with rigid and compliant environments but also, on the long run, with humans. Gradually, robots will require more advanced decisional autonomy, adaptability and ability to understand the intention of the ``other''.

In order to cope with scenarios and embodiments of increasing complexity, my research will advance the state of the art along the three directions considered above (modelling, estimation and control) with focus on three main topics: optimal control, compliance and uncertainty. 

First, optimal control, in the form of model predictive control, seems to be the natural prosecution of the current locally optimal solutions to whole-body control. Advances in this sense are foreseen both at the theoretical level (stability proof) and at the computational level (real-time optimal control of complete whole-body dynamics is still computationally cumbersome). 

Second, compliance and non-rigid interaction represent a major research topic that requires significant theoretical advances in considerations of the augmented degree of under-actuation and complexity which often results from compliant actuation. Again, successful application to whole-body motion control require stability proofs that represent a step forward with respect to the current state-of-the-art. 

Finally, following an idea already paved in my previous publications \citeALL{Berret20114354,Nori2015}, I will investigate on the role of stochasticity to represent the uncertainty that inevitably affect models of reality. It is indeed one of my guiding principles to consider no model as perfect \citeALL{Nori2015} and one of my ambitions to implement, on complex systems such as the iCub, control actions which optimally adapt to the uncertainties of acquired models \citeALL{Berret20133029}. 


%to design control actions which take into account the 
%
%propagate from models to estimation  and control  in the attempt . 

%models are likely to become stochastic, to include the implicit uncertainty in any model (i.e. modelling errors). 
%
%
%Consequently robots will be required to model, estimate and control the self (i.e. their own body) and the other (i.e. the environment) . In the long run, robots will be also required to interact with humans 
%
%
%Interaction in particular 
%
%
%
%This coordination necessarily passes through significant step forward in the way robots model, estimate and control the 
%
%advances adapting, estimating, exploiting 
%
%At first, I was responsible for developing a 
%
%
%
%The Optimal control as a way to obtain energetic efficiency (autonomy) by exploiting the intrinsic dynamics.
%Floating base as a way to achieve physical mobility. 
%
%Dynamics, in particular, capture forces during physical interaction, the core of my recent research activities. 
%
%
%
%real-time computational efficiency, accuracy in obtaining the control objectives, robustness with respect to modelling uncertainties and adaptability with respect to environmental changes. 
%
%
%
%
%Within this framework, my major contribution has been to bring my control theoretical expertise as a tool to approach relevant problems in the field of human and humanoid motor control. 
%
%
%
%
%Control theory guides my research activities, which can be clustered in 
% \bibliography{cv}
% To remove the references (works but gives an error)
\bibliographystyleSELECTED{unsrt}
\bibliographystyleOTHER{apalike}
\bibliographystyleALL{plain}
\bibliographystyleSOFTWARE{alpha}

\bibliographySELECTED{selectedBiblio}
\bibliographyALL{completeBiblio}
\bibliographySOFTWARE{softwareBiblio}
\bibliographyOTHER{otherBiblio}


\end{document}








